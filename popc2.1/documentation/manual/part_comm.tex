\chapter{Command Line Syntax}

\section{POP-C++ Compiler command}

\begin{verbatim} 
popcc [POP-C++ options] [other C++ options] sources...
POP-C++ options:
   -cxxmain:            Use standard C++ main (ignore POP-C++ 
                        initialization).
   -paroc-static:       Link with standard POP-C++ libraries 
                        statically.
   -paroc-nolib:        Avoid standard POP-C++ libraries from 
                        linking.
   -parclass-nointerface: Do not generate POP-C++ interface codes 
                          for parallel objects.
   -parclass-nobroker:    Do not generate POP-C++ broker codes 
                          for parallel objects.

   -object[=type]:      Generate parallel object executable 
                        (linking only)(type: std (default) or mpi)
   -popcpp:             POP-C++ parser
   -cpp=<preprocessor>: C++ preprocessor command
   -cxx=<compiler>:     C++ compiler
   -parocld=<linker>:   C++ linker (default: same as C++ compiler)
   -parocdir:           POP-C++ installed directory
   -noclean:            Do not clean temporary files
   -verbose:            Print out additional information

Environment variables change the default values used by POP-C++:
   POPC_LOCATION:  Directory where POP-C++ has been installed
   POPC_CXX:       The C++ compiler used to generate object code
   POPC_CPP:       The C++ preprocessor
   POPC_LD:        The C++ linker used to generate binary code
   POPC_PP:        The POP-C++  parser
\end{verbatim}
