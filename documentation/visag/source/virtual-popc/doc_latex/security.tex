\section{Security assumptions}
\label{sec:security}
This chapter aims to explain what we think are strengths and weaknesses on a security point of view in this project. The first section will focus on the strength as the second one will focus on the weaknesses. \s

\subsection{Security strengths}

\subsubsection{Application isolation}
An application is isolated from other application as a Worker Vm is started for each different application on a node. This means that an application can only disturb itself but not other running application.

\subsubsection{Admin VM isolation}
The Admin VM is totally isolated from the execution of the POP-C++ application. This VM do not accept other public key than the ones exchanged for the confidence link. An application should never be launched from an Admin VM. 

\subsubsection{Worker VM clean state}
A Worker VM is always reverted to its clean state before executing an application. Of course, this clean state depends on the person who created it during the installation of the node. 


\subsection{Security weaknesses}
\subsubsection{Node running the main}
Every POP-C++ application has a start point which is the main. The main must run on a node we will called the Main Node. The Main Node could be vulnerable has it will accept several public key from the worker used for this application. For the moment, we did not develop a perfect solution for this problem but some ideas are drawn in Section \ref{sec:future}.

\subsection{Possible security attacks}
This section aims to expose some of the possible attacks we found possible to happen on our middle-ware and the way to avoid them. 

\subsubsection{Sniffing the network}
By sniffing the network, the possible hacker could not do anything to our confidence network. Every connections are encrypted into a SSH Tunnel. 

\subsubsection{Hacking the Admin - VM}
An external person could hack into the Admin VM. As the Admin VM just acts as a routing point for the key exchange, and all the keys exchanged on the network are just public key, the hacker can only shut down the node and put a part of the POP-C++ network disable. 

\subsubsection{Enter the confidence network with a fake POP-C++}
A malicious developer could modify POP-C++ before installing it. This means that a different version of POP-C++ will enter a confidence network with standard POP-C++ VS. This kind of hack could disturb the good working of an execution. As the key transferred between the node are the public key, they do not pose a security issues if they are intercepted. 

\subsubsection{Make a POP Application with back doors}
A developer could make a POP Application that keep some process alive after its execution. As only one application is running in a Worker VM and these Worker VM are reverted at end, this kind of hack pose no problem to our middle-ware.

