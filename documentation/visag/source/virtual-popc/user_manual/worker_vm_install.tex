\pagebreak
\subsection{Set up the original Worker VM}
\label{lb:workervm}
As POP-C++ VS will clone workers as needed, we just need to set up the first "original" Worker VM. To set this particular worker VM, the following steps must be done : 

\textbf{\textit{WARNING:}} Read carfully the restriction imposed to the name, username and password. These restrictions are strict and cannot be worked around. 
\begin{itemize}
\item Create VM in vSphere (see Appendix \ref{app:createvm}). The VM name \textbf{must} have the suffix \textbf{\_worker1} (other suffixes are not supported for the moment).
\item Install an OS on the create VM.
\item Install necessary packages for POP-C++ (see Section \ref{lb:workerpackages})
\item Install VMware tools on the worker VM (see Appendix \ref{app:vmwaretools})
\item Configure SSH for the worker VM (see Section \ref{lb:workerssh})
\item Configure, Compile and Install POP-C++ Standard (see Section \ref{lb:workerpopc})
\item Create a snapshot of the worker VM (see Appendix \ref{app:snap})
\end{itemize}


\subsubsection{Necessary packages}
\label{lb:workerpackages}
To be able to run application using POP-C++ VS on the worker VM, we need to install some packages. Here are the list of the needed packages : 

\begin{itemize}
\item A C++ compiler
\item An SSH server
\item Zlib compression library
\end{itemize}

On a Debian/Ubuntu based OS, we can use the following command to install those packages.\s

\begin{lstlisting}
sudo apt-get install build-essential openssh-server zlib1g-dev
\end{lstlisting}

\subsubsection{Configure SSH server on the worker VM}
\label{lb:workerssh}
As POP-C++ VS uses the secure mode of POP-C++ to communicate between the nodes, we need to set up the SSH parameters. First, we need to generate the SSH public/private keys pair. Use the following command and keep the default values.\s

\begin{lstlisting}
ssh-keygen
\end{lstlisting}\s

We also need to let the SSH server accept host key without a strict check. For this, we need to modify the file \textbf{/etc/ssh/ssh\_config} and have the following line : \s

\begin{lstlisting}
StrictHostKeyChecking no
\end{lstlisting}

\subsubsection{Configure, Compile and Install POP-C++}
\label{lb:workerpopc}
On the Worker VM, we just need a standard version of POP-C++. Here are the commands to execute to configure and compile POP-C++ in its standard version.\s
\begin{lstlisting}
./configure
make
\end{lstlisting}

Once POP-C++ is compiled, we can install it. Use the following command : \s

\begin{lstlisting}
make install 
...
DO YOU WANT TO CONFIGURE  POP-C++ SERVICES? (y/n)
y
DO YOU WANT TO MAKE A SIMPLE INSTALLATION? (y/n) 
y
...
===============================
CONFIGURATION DONE!
===============================

IMPORTANT : Do not forget to add these lines to your .bashrc file or 
equivalent:

POPC_LOCATION=/home/visag/popc
PATH=$PATH:$POPC_LOCATION/bin:$POPC_LOCATION/sbin
\end{lstlisting}

\textit{NOTE:} Don't forget to add the lines in your .bashrc or equivalent file if you changed the default installation location otherwise the node will not run normally.