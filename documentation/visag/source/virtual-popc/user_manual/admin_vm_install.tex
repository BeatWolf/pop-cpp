\subsection{Set up the Admin VM}
\label{lb:adminvm}
The first VM to set up in the POP-C++ VS environement is an Admin VM. The Admin VM will interact with the ESXi hypervisor to manage the Worker VM. To set up this particular VM, the following steps must be done:

\begin{itemize}
\item Install ESXi on the computer (see Appendix \ref{app:esxi})
\item Create a VM in vSphere (see Appendix \ref{app:createvm})
\item Install an OS in the created VM. 
\item Install necessary packages for POP-C++ (see Section \ref{lb:packages})
\item Configure the DNS hostname (see Section \ref{lb:dns})
\item Configure the SSH server (see Section \ref{lb:sshserver})
\item Configure, Compile and Install POP-C++ (see Section \ref{lb:popc_install})
\end{itemize}

\subsubsection{Install necessary packages for POP-C++}
\label{lb:packages}
In order to install POP-C++ VS, we need to install some additional packages. The following packages need to be installed and the default values can be used : 

\begin{enumerate}
\item libvirt 0.8.5 or later (see Appendix \ref{app:libvirt})
\item VMware tools (see Appendix \ref{app:vmwaretools})
\item VMware CLI 4.1 or later (see Appendix \ref{app:vmwarecli})
\item VIX 10.1 or later (see Appendix \ref{app:vix})
\item A C++ compiler, Zlib
\end{enumerate}

For the point 5, the following command can be used on a Debian/Ubuntu based OS.\s

\begin{lstlisting}
sudo apt-get install build-essential zlib1g-dev
\end{lstlisting}





\subsubsection{Set the ESXi DNS hostname}
\label{lb:dns}
If your network does not have a DNS server or if the hostname of the ESXi platform is not registered in it, we need to add this hostname in the host file name of the Admin VM. On a linux based OS, we will modify the file located under \textbf{/etc/hosts} and add the following line (replace with your values):\s

\begin{lstlisting}
160.98.20.140	esxivisag01.sofr.hefr.lan
\end{lstlisting}





\subsubsection{Configure SSH Server}
\label{lb:sshserver}
As the Admin VM will communicate with worker VM and these Worker will be created dynamically, we need to configure the SSH server to avoid a strict host key identification. To do that, we need to modify the file \textbf{/etc/ssh/ssh\_config} and have the following line:\s

\begin{lstlisting}
StrictHostKeyChecking no
\end{lstlisting} \s

We need to generate the SSH public/private keys to be able to run Virtual Secure POP-C++. To generate the SSH keys, we need to run the following command (use the default parameters):\s
\begin{lstlisting}
ssh-keygen
\end{lstlisting}






\subsubsection{Configure and compile POP-C++ in its Virtual Secure version}
\label{lb:popc_install}
POP-C++ needs to be configured and compiled before its installation. During the configuration, we can choose the version of POP-C++ that we want to install. For the ViSaG project, we want to install POP-C++ VS. To have this version, use the following options with the configure script: \s

\begin{lstlisting}
./configure --enable-virtual --enable-secure
\end{lstlisting}\s

\textit{NOTE:} If you already have compiled POP-C++ before, use "make clean" to clean the workspace before compiling the new version.\s

Once the configuration process is done, we can compile POP-C++. For this, use the following command: \s

\begin{lstlisting}
make
\end{lstlisting}\s

If the make command exit without errors, POP-C++ is compiled in its VS version and ready to be installed. \s

To install POP-C++, we just need to launch the following command:\s

\begin{lstlisting}
make install
\end{lstlisting}\s

The installation script will ask us some question to configure our installation. Here are the different questions and the answer we could provide:\s

\textbf{Part 1:Select the right installation}
\begin{lstlisting}
DO YOU WANT TO CONFIGURE  POP-C++ SERVICES? (y/n)
y
...
DO YOU WANT TO MAKE A SIMPLE INSTALLATION? (y/n) 
n
\end{lstlisting}\s

\pagebreak
\textbf{Part 2: Configure the parameters of the node}\\
The first question of this part asks us to enter the "full qualified master host name". This means the hostname or IP address of the node which have the confidence link with this node. 
\begin{lstlisting}
Enter the full qualified master host name (POPC gateway):

\end{lstlisting}\s

Not used in POP-C++ VS for the moment (let blank)
\begin{lstlisting}
Enter the child node:

\end{lstlisting}\s

Enter the number of processors available on the whole node. 
\begin{lstlisting}
Enter number of processor available (default:1):

\end{lstlisting}\s

Set the number of jobs that can be concurrently executed on the node for all VM.
\begin{lstlisting}
Enter the maximum number of POP-C++ jobs that can run concurrently 
(default:100):
20
\end{lstlisting}\s

Set the available RAM for the whole node.
\begin{lstlisting}
Enter the available RAM for job execution in MB (default: 1024):
4096
\end{lstlisting}\s

Set the user you want to use to execute the jobs (usually the same as the one who installed POP-C++, let blank)
\begin{lstlisting}
Which local user you want to use for running POP-C++ jobs?

\end{lstlisting}\s


\begin{lstlisting}
CONFIGURE THE RUNTIME ENVIRONMENT
Enter the script to submit jobs to the local system:

Communication pattern:

\end{lstlisting}\s

\pagebreak
\textbf{Part3: Setting up the virtual environment}\\
It's time to set up the virtual environment for POP-C++. We need the information about the ESXi platform and the worker VM. If the worker VM is not set up yet, these information can be edited later in the file POPC\_LOCATION/etc/virtual.conf.
\begin{lstlisting}
SETTING UP VIRTUAL ENVIRONMENT INFORMATION NOW
ESX(i) hypervisor IP address : e.g. 160.98.20.140
160.98.20.141
ESX(i) user name with admin rights (default: root)
root
ESX(i) password:
*****
ESX(i) datastore name (default: datastore1)

ESX(i) maximum worker (default: 4)

ESX(i) worker name (default: popc_worker_guest1)
visag04_worker1
ESX(i) worker OS username
visag
ESX(i) worker OS password
*****
ESX(i) clean snapshot name (default: popc_clean)
popc_clean
\end{lstlisting}\s





\textbf{Part 4: runtime environment variables}\\
We can set up some environment variables to be used in the runtime of POP-C++. No particular variables needs to be set for POP-C++ VS.
\begin{lstlisting}
SETTING UP RUNTIME ENVIRONMENT VARIABLES
Enter variable name:

Enter variable value:

\end{lstlisting}\s

\pagebreak
\textbf{Part 5: Installation end}\\
It's important to generate the startup script because it will be used to run the POP-C++ Global Services on the Admin VM.
\begin{lstlisting}
===============================
CONFIGURATION POP-C++ SERVICES COMPLETED
===============================
Do you want to generate the POP-C++ startup scripts? (y/n)
y
Enter the service port[2711]:

Enter the domain name:

Enter the temporary directory for intermediate results:

===============================
CONFIGURATION DONE!
===============================

IMPORTANT : Do not forget to add these lines to your .bashrc file or 
equivalent:

POPC_LOCATION=/home/visag/popc
PATH=$PATH:$POPC_LOCATION/bin:$POPC_LOCATION/sbin
\end{lstlisting}\s




\subsubsection{Run the POP-C++ services on the Admin VM}
After the preparation and installation of POP-C++ VS on the Admin VM, we can now start the Global Services on the node. Use this command to start the POP-C++ Global Services : 
\begin{lstlisting}
SXXpopc start
\end{lstlisting}\s

You will see the followings lines if the startup is successful (the IP addresses and ports will be different):\s
\begin{lstlisting}
Starting POP-C++ [Virtual Secure Version] Global Services
VSPSN Started [socket://160.98.21.195:49391]
VPSM Started [socket://160.98.21.195:37958]
POPCloner Started [socket://160.98.21.195:52035]
VSJM created [socket://160.98.21.195:2711]
\end{lstlisting}\s

If an exception occurs at start up, please check one of the following log files to find the problem: 

\begin{itemize}
\item /tmp/popc\_node\_log
\item /tmp/popc\_security\_log
\item /tmp/popc\_clone\_log
\end{itemize}

\textit{NOTE:} The /tmp directory is the default log files directory you may have changed it with another directory
 