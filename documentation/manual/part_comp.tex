\chapter{Compiling and Running}

%\todo{Here should be an explanation about the chapter.
% It should also say that an example is given.}


%%%%%%%%%%%%%%%%%%%%%%%%%%%%%%%%%%%%%%%%%%%%%%%%%%%%%%%%%%%%%%%%%%%%%%%%%%%
\section{Compilation}
%%%%%%%%%%%%%%%%%%%%%%%%%%%%%%%%%%%%%%%%%%%%%%%%%%%%%%%%%%%%%%%%%%%%%%%%%%%

%The POP-C++ language is an extension of C++. It incorporates its own compiler which translates the POP-C++ code to ANSI C++ code before compiling it. 
The POP-C++ compiler generates a main executable and several object executables. 
The main executable provides a starting point to the application and object executables are loaded and started by the POP-C++ runtime system whenever a parallel object is created. The compilation process is illustrated in figure \ref{fig_compiler}.

%the POP-C++ service libraries that
%provide APIs for accessing various runtime services such as
%communication, resource discovery and object allocation, etc.; and an
%ANSI C++ compiler to generate binary executables from the C++ code and
%the service libraries. 

\figureps{fig_compiler}{140mm}
{POP-C++ compilation process}

The POP-C++ compiler contains a parser which translates the
code to ANSI C++ code. Service libraries provide APIs that manages communication,
resource discovery and object allocation, etc. An ANSI C++ compiler finally
generates binary executables.


\section{Example Program}

We will see in this section how to write a simple POP-C++ program. 

\subsection{Programming}
\subsubsection{integer.ph}

Figure \ref{fig_integer_ph} shows the declaration of a parallel class in a
POP-C++ header. From the language aspect, this part contains the major
differences between POP-C++ and C++. However, as the example shows,
a POP-C++ class declaration is similar to a C++ class declaration with the addition of some new keywords. A parallel class consists of
constructors (lines 3 and 4), destructor (optional), interfacing methods
(\texttt{public}, lines 5-7), and a data attribute (\texttt{private},
line 9).

\begin{figura}{fig_integer_ph}{File \texttt{integer.ph}}%
\vspace{-4mm}%
\begin{verbatim}
 1: parclass Integer {
 2: public :
 3:     Integer(int wanted, int minp) @{ od.power(wanted, minp); };
 4:     Integer(POPString machine) @{ od.url(machine);};
 5:     seq async void Set(int val);
 6:     conc int Get();
 7:     mutex void Add(Integer &other);
 8: private :
 9:     int data;
10: };
\end{verbatim}
\end{figura}

In the figure \ref{fig_integer_ph}, the programmer defines a parallel
class called \texttt{Integer} starting with the keyword
\texttt{parclass} (line 1). Two constructors (lines 3 and 4) of
\texttt{Integer} are both associated with two object descriptors which
reside right after the argument declaration, between \texttt{@\{...\}}.
The first object descriptor (line 3) specifies the parameterized high
level requirement of resource (i.e. computing power). The second object
descriptor (line 4) is the low-level description of the location of
resource on which the object will be allocated.


The invocation semantics are defined in the class declaration by putting
corresponding keywords (\texttt{sync}, \texttt{async}, \texttt{mutex},
\texttt{seq}, \texttt{conc}) in front of the method declaration. In the
example of figure \ref{fig_integer_ph}, the \texttt{Set()} method (line
5) is sequential asynchronous, the \texttt{Get()} method (line 6) is
concurrent and the \texttt{Add()} method (line 7) is mutual exclusive
execution. Although it is not shown in the example the user can also
use standard C++ features such as \texttt{virtual}, \texttt{const}, or
inheritance with the parallel class.


\subsubsection{integer.cc}
\begin{figura}{fig_integer_cc}{File \texttt{integer.cc}}%
\vspace{-4mm}%
\begin{verbatim}
 1: #include "integer.ph"
 2: 
 3: Integer::Integer(int wanted, int minp) {}
 4: Integer::Integer(POPString machine) {}
 5: 
 6: void Integer::Set(int val) {
 7:     data = val;
 8: }
 9: 
10: int Integer::Get() {
11:     return data;
12: }
13: 
14: void Integer::Add(Integer &other) {
15:     data = other.Get();
16: }
17: 
18: @pack(Integer);
\end{verbatim}
\end{figura}

The implementation of the parallel class \texttt{Integer} is shown in
figure \ref{fig_integer_cc}. This implementation does not contain any
invocation semantics and looks similar to a C++ code, except at line 18
where we provide a directive \texttt{@pack} to tell the POP-C++ compiler
the place to generate the parallel object executable for
\texttt{Integer} (see section \ref{sec:pack} for the \texttt{pack} directive).


\subsubsection{main.cc}
\begin{figura}{fig_main_cc}{File \texttt{main.cc}}%
\vspace{-4mm}%
\begin{verbatim}
 1: #include "integer.ph"
 2:
 3: int main(int argc, char **argv)
 4: {
 5:     try {
 6:         Integer o1(100, 80), o2("localhost");
 7:         o1.Set(1);
 8:         o2.Set(2);
 9:         o1.Add(o2);
10:         printf("Value=%d\n", o1.Get());
11:     }
12:     catch (POPException *e) {
13:         printf("Object creation failure\n");
14:         e->Print();
15:         return -1;
16:     }
17:     return 0;
18: }
\end{verbatim}
\end{figura}


The main POP-C++ program in figure \ref{fig_main_cc} looks exactly like a
C++ program. Two parallel objects of type \texttt{Integer}, \texttt{o1}
and \texttt{o2}, are created (line 6). The object \texttt{o1} requires a
resource with the desired performance of 100MFlops although the minimum
acceptable performance is 80MFlops. The object \texttt{o2} will
explicitly specify the resource location (localhost). 

After the object
creations, the invocations to methods \texttt{Set()} and \texttt{Add()}
are performed (line 7-9). The invocation of \texttt{Add()} method shows
an interesting property of the parallel object: the object \texttt{o2}
can be passed from the \texttt{main} program to the remote method
\texttt{Add()} of parallel object \texttt{o1}. 

Lines 12-15 illustrate
how to handle exceptions in POP-C++ using the keyword pair \texttt{try}
and \texttt{catch}. Although \texttt{o1} and \texttt{o2} are distributed
objects but the way to handle the remote exceptions is the same as in
C++.

\subsection{Compiling}

We generate two executables: the main program (\texttt{main}) and the
object code (\texttt{integer.obj}). POP-C++ provides the command
\texttt{popcc} to compile POP-C++ source code. To compile the main
program we use the following command:

\vspace{3mm}
\begin{verbatim}
    popcc -o main integer.ph integer.cc main.cc
\end{verbatim}

\subsection{Compile the object code}

Use \texttt{popcc} with option \texttt{-object} to generate the object
code:

\vspace{4mm}
\begin{verbatim}
    popcc -object -o integer.obj integer.ph integer.cc
\end{verbatim}
\vspace{4mm}

You can note that we have compiled the declaration of the parallel class
\texttt{integer.ph} explicitly. The user can also generate intermediate
code \texttt{.o} that can be linked using a C++ compiler by using the \texttt{--c} option (compile only) with \texttt{popcc}.

\subsubsection{Compilation for several parclasses with dependencies}

The compilation is a little bit more difficult for more complex applications 
using several different parallel classes. This is the case, for example,
when the main program calls methods from objects of different parallel classes
or when there is a chain of dependencies between the main program and several
parallel classes as illustrated on figure \ref{multi_class}.

\figureps{multi_class}{120mm}
{Parclasses with dependencies}

\begin{figura}{fig_dependency_cc}{How to compile applications with dependencies}%
\vspace{-4mm}%
\begin{verbatim}
	popcc -object -o myobj4.obj myobj4.ph myobj4.cc
	popcc -c -parclass-nobroker myobj4.ph

	popcc -object -o myobj3.obj myobj3.ph myobj3.cc myobj4.stub.o
	popcc -c -parclass-nobroker myobj3.ph
	
	popcc -object -o myobj2.obj myobj2.ph myobj2.cc myobj3.stub.o
	popcc -c -parclass-nobroker myobj2.ph 

	popcc -object -o myobj1.obj myobj1.ph myobj1.cc myobj2.stub.o

	popcc -o main main.cc myobj1.ph myobj1.cc myobj2.stub.o
\end{verbatim}
\end{figura}

Since each class contains some internal POP-C++ classes such as the
\texttt{interface} or the \texttt{broker} classes, the compilation must avoid
to create multiple definitions of these classes. An easy way to avoid this is to
begin the compilation with the last class of the chain (the class \texttt{myobj4}
on figure \ref{multi_class}) and then to compile each parallel class in reverse
order. To compile any class in the chain we needs the parallel classe which is
directly after the one we are compiling in the chain of dependency.
When compiling a parallel classe without generating the executable code
(option \texttt{-c}), the POP-C++ compiler generates a relocatable object file
called className\texttt{.stub.o}. In addition the POP-C++ compiler has an option
called \texttt{-parclass-nobroker} which allows to generate relocatable code
without internal POP-C++ classes.
The way to compile a POP-C++ application with dependencies as illustrated on figure \ref{multi_class} is shown on figure \ref{fig_dependency_cc}.

The source code of this example can be found in the \texttt{examples} directory
of the POP-C++ distribution and on the POP-C++ web site 
(\texttt{http://gridgroup.hefr.ch/popc}).



\subsection{Running}

To execute a POP-C++ application we need to generate the object map file which contains
the list of all parallel classes used by the application. For each parallel
class we need to indicate for which architecture the compilation has been done
and the location of the file (object file).

With POP-C++ it is possible to get this information by executiong the object file
with the option \texttt{-listlong}.

Example for the \texttt{Interger} parallel class:
\vspace{4mm}
\begin{verbatim}
    ./integer.obj -listlong
\end{verbatim}
\vspace{4mm} 

To generate the object map file we simply redirect the output to the object map file:
\vspace{4mm}
\begin{verbatim}
    ./integer.obj -listlong > obj.map
\end{verbatim}
\vspace{4mm}

The object map file contains all mappings between object name,
platform and the executable location. In our example we have compiled on Linux machines and the object map file looks like this:

\vspace{4mm}
\begin{verbatim}
    Integer i686-pc-Linux /home/myuser/popc/test/integer/integer.obj
\end{verbatim}
\vspace{4mm}

If you also compile the object code for another platform (e.g. Solaris), simply
add a similar line to \texttt{obj.map}. The executable location can also be an
absolute path or an URL (HTTP or FTP). 

We can now run the program using the command \texttt{popcrun}:

\vspace{3mm}
\begin{verbatim}
    popcrun obj.map ./main
\end{verbatim}
\vspace{4mm}

Figure \ref{fig_integer_run} shows the execution of \texttt{Integer::Add()}
method on line 4 in figure \ref{fig_integer_cc} of the example. The system
consists of three running processes: the \texttt{main}, object
\texttt{o1} and object \texttt{o2}. The \texttt{main} is started by the
user. Objects \texttt{o1} and \texttt{o2} are created by \texttt{main}.
Object \texttt{o2} and the \texttt{main} program run on the same machine
although they are in two separate memory address spaces; object \texttt{o1} runs
on a remote machine. The \texttt{main} invokes the \texttt{o1.Add()}
with the interface \texttt{o2} as an argument. Object \texttt{o1} will
then connect to \texttt{o2} automatically and invoke the method
\texttt{o2.Get()} to get the value and to add this value to its local
attribute \texttt{data}. POP-C++ system manages all object interactions
in a transparent manner to the user.

\figureps{fig_integer_run}{100mm}
{An execution example}
