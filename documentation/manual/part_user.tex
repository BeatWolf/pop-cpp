\chapter{User Manual}

%%%%%%%%%%%%%%%%%%%%%%%%%%%%%%%%%%%%%%%%%%%%%%%%%%%%%%%%%%%%%%%%%%%%%%%%%%%
\section{Introduction}
%%%%%%%%%%%%%%%%%%%%%%%%%%%%%%%%%%%%%%%%%%%%%%%%%%%%%%%%%%%%%%%%%%%%%%%%%%%


The POP model (see chapter \ref{chap:model}) is a suitable programming model for
large heterogenous distributed environments but it should also remain
as close as possible to traditional object oriented programming. 
Parallel objects of the POP model generalize sequential objects, 
keep the properties of object oriented programming (data encapsulation,
inheritance and polymorphism) and add new properties.

The POP-C++ language is an extension of C++ that implements the POP model.
Its syntax remains as close as possible to standard
C++ so that C++ programmers can easily learn it and existing
C++ libraries can be parallelized without much effort.
Changing a sequential C++ application into a distributed parallel
application is rather straightforward.

Parallel objects are created using parallel classes. Any object that
instantiates a parallel class is a parallel object and can be executed
remotely. To help the POP-C++ runtime to choose a remote machine
to execute the remote object, programmers can add object description
information to each constructor of the parallel object. In order to
create parallel execution, POP-C++ offers new semantics for method
invocations. These new semantics are indicated thanks to five new
keywords. Synchronizations between concurrent calls are sometimes
necessary, as well as event handling; the standard POP-C++ library supplies
some tools for that purpose. This chapter describes the syntax of the
POP-C++ programming language and presents main tools available in the
POP-C++ standard library.



%%%%%%%%%%%%%%%%%%%%%%%%%%%%%%%%%%%%%%%%%%%%%%%%%%%%%%%%%%%%%%%%%%%%%%%%%%%
\section{Parallel Objects}
%%%%%%%%%%%%%%%%%%%%%%%%%%%%%%%%%%%%%%%%%%%%%%%%%%%%%%%%%%%%%%%%%%%%%%%%%%%


POP-C++ parallel objects are a generalization of sequential objects. Unless the term
\textbf{sequential object} is explicitly specified, a parallel object is
simply referred to as an object in teh rest of this chapter.




% This is done by
%replacing the \texttt{class} keyword by the new keyword
%\texttt{parclass}. Any object that instantiates a parallel class is a
%parallel object and can be executed remotely. In order to help the
%POP-C++ runtime to choose a remote machine for the remote object
%execution, programmers can add \emph{object description} information
%(OD) to each constructor.
%
%When creating remote objects two situations can occur: the remote object
%immediately starts to execute some code (active object semantic) or the
%remote object only executes code when there are method invocations
%(static object semantic). The POP model is based on \emph{static object
%semantic} because this semantic is closer from the one of traditional
%sequential OO programming. In order to create parallel execution we add
%new semantics to method invocations. This new semantic are indicated
%through five new keywords: \texttt{synch}, \texttt{async}, \texttt{seq},
%\texttt{mutex} and \texttt{conc}. Together with object descriptions, the
%six new keywords presented above constitute the core of the POP-C++
%extension to C++. The rest of the C++ syntax remains identical and
%especially class implementation are realized using pure C++ syntax.


\subsection{Parallel Class} \label{sec:class:dec}

Developing POP-C++ programs mainly consists of designing and
implementing parallel classes. The declaration of a parallel class
begins with the keyword \texttt{parclass} followed by the class name and
the optional list of derived parallel classes separated by commas:

\begin{verbatim}
    parclass ExampleClass {
        /* methods and attributes */
        ...
    };
\end{verbatim}

or

\begin{verbatim}
    parclass ExampleClass: BaseClass1, BaseClass2 {
        /* methods and attributes */
        ...
    };
\end{verbatim}

As in the C++ language, multiple inheritance and polymorphism are supported
in POP-C++. A parallel class can be a stand-alone class or it can be
derived from other parallel classes. Some methods of a
parallel class can be declared as overridable (virtual methods).

Parallel classes are very similar to standard C++ classes. Nevertheless,
same restrictions applied to parallel classes. 

\begin{petitem}

\item All data attributes are protected or private;

\item The objects do not access any global variable;

\item There are no programmer-defined operators;

\item There are no methods that return memory address references.

\end{petitem}

These restrictions are not a major issue in object-oriented programming
and in some cases they can improve the legibility and the clearness of
programs. The restrictions can be mostly worked around by adding
\texttt{get()} and \texttt{set()} methods to access data attributes and
by encapsulating global data and shared memory address variables in
other parallel objects.


\subsection{Creation and Destruction}

The object creation process consists of several steps: locating a
resource satisfying the object description (resource discovery),
transmitting and executing the object code, establishing the
communication, transmitting the constructor arguments and finally
invoking the corresponding object constructor. Failures on the object
creation will raise an exception to the caller. Section
\ref{sec:exceptions} will describe the POP-C++ exception mechanism.

As a parallel object can be accessible concurrently from multiple
distributed locations (shared object), destroying a parallel object
should be carried out only if there is no other reference to the object. 
POP-C++ manages parallel objects' life time by an internal reference
counter. A null counter value will cause the object to be physically
destroyed.

Syntactically, the creation and the destruction of a parallel object are
identical to those of C++. A parallel object can be implicitly created
by just declaring a variable of the type of parallel object on stack or
using the standard C++ \texttt{new} operator. When the execution goes
out of the current stack or the \texttt{delete} operator is used, the
reference counter of the corresponding object is decreased.


\subsection{Parallel Class Methods}

Like sequential classes, parallel classes contain methods and attributes.
Method can be public, or private while attribute must be either protected
or private. For each method, the programmer should define the invocation
semantics. These semantics, described in section \ref{sec:inv:sem}, are
specified by two keywords, one for each side:

\begin{petitem}

\item Interface side:

	\begin{petitem}

	\item \texttt{sync}: Synchronous invocation. This is the
		default value. For example: \\
		\texttt{sync void method1();}

	\item \texttt{async}: Asynchronous invocation. For example:\\
		\texttt{async void method2();}

	\end{petitem}

\item Object side:

	\begin{petitem}

	\item \texttt{seq}: Sequential invocation. This is the default
		value. For example:\\
		\texttt{seq void method1();}

	\item \texttt{mutex}: Mutex invocation. For example:\\
		\texttt{mutex int method2();}

	\item \texttt{conc}: Concurrent invocation. For example:\\
		\texttt{conc float method3();}

	\end{petitem}

\end{petitem}

The combination of the interface and the object-side semantics defines
the overall semantics of a method. For instance, the following
declaration defines an synchronous concurrent method that returns an
integer number:

\begin{verbatim}
    sync conc int myMethod();
\end{verbatim}


%The programmer can add optional argument descriptors to each argument to
%provide additional information for POP-C++ to marshal data. Arguments
%can be for input or output or both.

%A custom marshal and demarshal procedure can be supplied. In the
%current version of POP-C++, all non-structured C++ data types are
%automatically marshaled. For structured data types, the programmer
%should also specify the function to marshal data by an optional
%descriptor. 

%If an argument is an array, it is also necessary that the
%programmer provide a hint on the number of elements. 

Figure \ref{fig:ex:sort} contains an example of a method \texttt{sort()} that
has two arguments: an array of integer data (for input and output) and
its (integer) size.

\begin{figura}{fig:ex:sort}{Array argument example}%
\vspace{-4mm}%
\begin{verbatim}
    parclass Table {
        ...
        void sort([in, out, size=n] int *data, int n);
        ...
    };

        /* main program */
        ...
        Table sales;
        int amount[10];
        sales.sort(amount, 10);
        ...
\end{verbatim}
\end{figura}

%\todo{This example should include a marshall function.}



\subsection{Object Description}

Object descriptions are used to describe the resource requirements
for the execution of the object. Object descriptions are
declared along with parallel object constructor statements. Each
constructor of a parallel object can be associated with an object
description that resides directly after the argument declaration.
The syntax of an object descriptor is as follows::

\texttt{~~~~@\{}$expressions$\texttt{\}}

An object description contains a set of resource requirement
expressions. All resource requirement expressions are separated by
semicolons and can be any of the following:

\texttt{~~~~od.}$res_N$\texttt{(}$exact$\texttt{);}

\texttt{~~~~od.}$res_N$\texttt{(}$exact$\texttt{, }$lbound$\texttt{);}

\texttt{~~~~od.}$res_S$\texttt{(}$resource$\texttt{);}

\texttt{~~~~}$res_N$ := \texttt{power}$~|~$
	\texttt{memory}$~|~$
	\texttt{network}$~|~$
	\texttt{walltime}

\texttt{~~~~}$res_S$ := \texttt{protocol}$~|~$
	\texttt{encoding}$~|~$\texttt{url}

Both $exact$ and $lbound$ terms are numeric expressions, and
$resource$ is a null-terminated string expression. The semantics of those
expressions depend on the resource requirement specifier (the keyword
corresponding to $res_N$ or $res_S$).
The $lbound$ term is only used in non-strict object descriptions, to
specify the lower bound of the acceptable resource requirements.

The current implementation allows indicating resources requirement in
terms of:

\begin{petitem}

\item Computing power (in Mflops), keyword \texttt{power}

\item Memory size (in MB), keyword \texttt{memory}

\item Bandwidth (in Mb/s), keyword \texttt{network}

\item Location (host name or IP address), keyword \texttt{url}

\item Protocol (\texttt{"socket"} or \texttt{"http"}), keyword \texttt{protocol}

\item Data encoding (\texttt{"raw"}, \texttt{"xdr"}, \texttt{"raw-zlib"}
	or \texttt{"xdr-zlib"}), keyword \texttt{encoding}

\end{petitem}

An example of object description is given in the figure \ref{fig:ex:od}.
There, the constructor for the parallel object \texttt{Bird} requires
the computing power of \texttt{P} Mflops, the desired memory space of
100MB (having 60MB is acceptable) and the communication protocol is
socket or HTTP (socket has higher priority).


\begin{figura}{fig:ex:od}{Object descriptor example}%
\vspace{-4mm}%
\begin{verbatim}
    parclass Bird
    {
    public:
        Bird(float P) @{ od.power(P);
                         od.memory(100,60);
                         od.protocol("socket http"); };
        ...
    };
\end{verbatim}
\end{figura}


Object descriptors are used by the POP-C++ runtime system to find a
suitable resource for the parallel object. Matching between object
descriptors and resources is carried out by a multi-layer filtering
technique: first, each expression (item) in every object descriptor will
be evaluated and categorized (e.g., power, network, memory). Then, the
matching process consists of several layers; each layer filters single
category within object descriptors and performs matching on that
category. Finally, if an object descriptor pass all filters, the object
is assigned to that resource.

If no suitable resource is found to execute the objet then an exception
is raised (see section \ref{sec:exceptions}).


\subsection{Data marshaling} \label{sec:marshaling}

When calling remote methods, the arguments must be transferred to the
object being called (the same happens for returned values). In order to
operate with different memory spaces and different architectures, data is marshaled
into a standard format prior to be send to remote objects. All data passed
is serialized (marshalled) at the caller side and deserialized (demarshaled)
at the callee side.

Programmers can help the POP-C++ compiler to generate efficient code by
optionally specifying which arguments to transfer. This is done using an
argument information block that can contain the directives \texttt{in}
(for input), \texttt{out} (for output), or both. The argument
information block should appear between braces (\texttt{[} and
\texttt{]}), right before each argument declaration.
Only input arguments are transferred from the caller to the remote object. 
Output arguments will only be transferred back to the
caller for a synchronous method invocation. Without those
directives, in the current implementation of POP-C++ the following rules
are applied:

\begin{petitem}

    \item If the method is asynchronous, arguments are input-only.

    \item If the method is synchronous:

	\begin{petitem}

	    \item Constant and passing-by-value arguments are
		input-only.

	    \item Other arguments are considered as both
		input and output.

	\end{petitem}

\end{petitem}


POP-C++ automatically marshal/demarshal all the basic types of C++
(\texttt{int, float, char,} ... etc.). For arrays arguments programmers
have to explicitly supply the number of elements the array contains.
This is done using the directive \texttt{size} in the argument
information block.
 
Void pointers (\texttt{void*}) cannot be used as arguments of parallel object methods.

For structured data, the programmer must supply a marshalling and
demarshalling function through the directive \texttt{proc=}\emph{<function
name>} in the argument information block (see subsection \ref{sbsec:marshallingData}).

Finally to pass sequential objects as arguments to a \texttt{parclass} method,
programmers must derive their classes from a POP-C++ system class called
\texttt{POPBase} and implement the virtual method \texttt{Serialize} (see
subsection \ref{sbsec:marshallingObject}).

The POP-C++ system library provides two classes to support user specific
marshalling/demarshalling functions: \texttt{POPBuffer} representing a
system buffer that store marshalled data and \texttt{POPMeemSpool}
representing the temporary memory spool that can be used to allocate
temporary memory space for method invocation. The interfaces of these
two classes are discussed bellow:

\begin{verbatim}
class  POPBuffer
{
public:
   void Pack(const Type *data, int n);
   void UnPack(Type *data, int n);
};

class POPMemSpool
{
public:
   void *Alloc(int size);
};

\end{verbatim}


The \texttt{POPBuffer} class contains a set of \texttt{Pack}/\texttt{UnPack} methods for
all simple data types \texttt{Type} (\texttt{char},\texttt{bool},\texttt{int},\texttt{float}, etc.). \texttt{Pack}
is used to marshal the array of \texttt{data} of size \texttt{n}.  \texttt{Pack} is used
to demarshal the \texttt{data} from the received buffer. 

\subsection{Marshalling Sequential Objects} \label{sbsec:marshallingObject}

To be able to pass sequential objects as arguments to a \texttt{parclass} method,
programmers must derive their classes from a POP-C++ system class called
\texttt{POPBase} and implement the virtual method \texttt{Serialize}.
The interface of \texttt{POPBase} is described as following.

\begin{verbatim}
class POPBase
{
 public:
  virtual void Serialize(POPBuffer &buf, bool pack);
};
\end{verbatim}

The method \texttt{Serialize} requires two arguments: the \texttt{buf} that stores the
object data and flag \texttt{pack} specifying if it is to serialize data into the
buffer or to deserialize data from the buffer.


  \begin{figura}{fig:ex:marshalobj}{Marshalling an object}%
\vspace{-4mm}%
\begin{verbatim}
    class Speed: public POPBase {
       public:
         Speed();
         virtual void Serialize(POPBuffer &buf, bool pack);
  
         float *val;
         int count;
    };
    void Speed::Serialize(POPBuffer &buf, bool pack) {
      if (pack) {
        buf.Pack(&count,1);
        buf.Pack(val, count);
      }
      else {
        if (val!=NULL) delete [] val;
        buf.UnPack(&count,1);
        if (count>0) {
          val=new float[count];
          buf.UnPack(val, count);
        }
        else val=NULL;
      }
    }

    parclass Engine {
        ...
        void accelerate(const Speed &data);
        ...
    };
\end{verbatim}
\end{figura}

Figure \ref{fig:ex:marshalobj} shows an example of marshalling \texttt{Speed}
class compared to Fig.\ref{fig:ex:marshal}. Instead of specifying the
marshalling function, the programmer implements the
method \texttt{Serialize} of the \texttt{POPBase} class.

\subsection{Marshalling Data Structures} \label {sbsec:marshallingData}

For marshalling/demarshalling complex structures which are not objects, such as
\texttt{struct} of C++, programmers need to indicate which function to use for
marshalling/demarshalling the structure. In addition it is necessary to
allocate temporary memory to store the structure to be sent. This memory
space will be then freed automatically by the system after the invocation is
finished. POP-C++ provides a class \texttt{POPMemSpool} with the method
\texttt{Alloc} to do this temporary memory allocation as well as a way to
indicate, when calling a method, the function to use for
marshalling/demarshalling an argument (\texttt{proc=}).
  

\begin{figura}{fig:ex:marshal}{Marshalling a structure}%
\vspace{-4mm}%
\begin{verbatim}
    struct Speed {
        float *val;
        int count;
    };

    void marsh(POPBuffer &buffer, Speed &data, int count,
                            int flag, POPMemSpool *tmpmem) {
      if (flag & FLAG_MARSHAL)
      {
        buffer.Pack(&data.count,1);
        buffer.Pack(data.val, data.count);
      }
      else
      {
        buffer.UnPack(&data.count,1);
        //performing temporary allocation before calling UnPack
        data.val=(float *)tmpmem->Alloc(data.count*sizeof(float)); 
        buffer.UnPack(data.val, data.count);
      }
    }

    parclass Engine {
        ...
        void accelerate([proc=marsh] const Speed &data);
        ...
    };
\end{verbatim}
\end{figura}

Figure \ref{fig:ex:marshal} shows an example of data structure marshalling
in POP-C++.  In this example, the programmer provides the function
\texttt{marsh()} for marshalling/demarshalling the argument \texttt{data}
of method \texttt{accelerate()} of parallel class \texttt{Engine}.
The programmer provided marshalling function
\texttt{marsh()} takes five arguments:

\begin{petitem}

	\item \texttt{buffer}: a buffer to marshal data into or
		demarshal from.

	\item \texttt{data}: the data structure to be marshalled or
		demarshalled, passed by reference.

	\item \texttt{count}: the number of elements to marshal or
		demarshal.

	\item \texttt{flag}: a bit mask that specifies where this
		function is called (marshalling or demarshalling, interface side or server-side). 

	\item \texttt{tmpmem}: a temporary memory spool (POPMemspool). 

\end{petitem}


The marshalling function should be implemented in such a way that, when
called to marshal, it packs all relevant fields of \texttt{data} into
\texttt{buffer}. Likewise, when called to unmarshall, it should unpack
all \texttt{data} fields from \texttt{buffer}. The \texttt{buffer} has
two methods, overloaded for all scalar C++ types, to be used to pack and
unpack data. These methods are \texttt{Pack()} and \texttt{UnPack()}
respectively. Both methods are used with the number of items to pack,
one for scalars, more than one for vectors.


\texttt{data} is passed to the marshalling function by reference, so the
function can modify it if necessary. Also, if \texttt{data} is a vector
(not the case shown in the example), the argument \texttt{count} will be
greater than one.

A bit mask (\texttt{flag}) is passed to the marshalling function to
specify whether it should marshal or demarshal data. The bit mask
contains several bit fields, and if the bit \texttt{FLAG\_MARSHAL} is
set, the function should marshal data. Otherwise, it
should demarshal data. If the bit \texttt{FLAG\_INPUT} is set,
the function is called at the interface side. Otherwise,
it is called at the object-server side.

The last argument of the function (\texttt{tmpmem}) should be only used to 
allocate temporary memory space. In the example, the \text{Speed}
structure contains an array \texttt{val} of \texttt{count} elements. At the
object-side, before unpacking \texttt{val}, we need to perform temporary
memory allocation using the memory spool interface provided by \texttt{tmpmem}.  
 
%\todo{the text above is a lie. We should describe tmpmem's methods.}


%%%%%%%%%%%%%%%%%%%%%%%%%%%%%%%%%%%%%%%%%%%%%%%%%%%%%%%%%%%%%%%%%%%%%%%%%%%%
\section{Object Layout} \label{sec:pack}
%%%%%%%%%%%%%%%%%%%%%%%%%%%%%%%%%%%%%%%%%%%%%%%%%%%%%%%%%%%%%%%%%%%%%%%%%%%%

A POP-C++ application is build using several executable files. One of
them is the main program file, used to start running the application.
Other executable files contain the implementations of the parallel
classes for a specific platform. An executable file can store the
implementation of one or several parallel objects. Programmers can help
the POP-C++ compiler to group parallel objects into a single executable
file by using the directive \texttt{@pack()}.

\begin{figura}{fig:pack}{Packing objects into an executable file}%
\vspace{-4mm}%
\begin{verbatim}
    Stack::Stack(...) {
        ...
    }
    Stack::push(...) {
        ...
    }
    Stack::pop(...) {
        ...
    }

    @pack(Stack, Queue, List)
\end{verbatim}
\end{figura}

All POP-C++ objects to be packed in a single executable file should be
included as arguments of the \texttt{@pack()} directive. It is required
that among the source codes passed to the compiler, exactly one source
code must contain \texttt{@pack()} directive. Figure \ref{fig:pack}
shows an example with a file containing the source code of a certain
class \texttt{Stack}, and a \texttt{@pack()} directive requiring that in
the same executable file should be packed the executable code for the
classes \texttt{Stack}, \texttt{Queue} and \texttt{List}.



%%%%%%%%%%%%%%%%%%%%%%%%%%%%%%%%%%%%%%%%%%%%%%%%%%%%%%%%%%%%%%%%%%%%%%%%%%%
\section{Class Library}
%%%%%%%%%%%%%%%%%%%%%%%%%%%%%%%%%%%%%%%%%%%%%%%%%%%%%%%%%%%%%%%%%%%%%%%%%%%


Alongside with the compiler, POP-C++ supplies a class library. This
library basically offers classes for dealing with synchronizations and
exceptions. These library classes are described in this section.


\subsection{Synchronization}

POP-C++ provides several method invocation semantics to control the
level of concurrency of data access inside each parallel object. 
Communication between threads using shared attributes is straightforward
because all threads on the same object share the same memory address
space. When concurrent invocations happen, it is possible that they
concurrently access an attribute, leading to errors. The programmer
should verify and synchronize data accesses manually. To deal with this
situation, it could be necessary to synchronize the concurrent threads of
execution.

\begin{figura}{fig:synchro}{The POPSynchronizer class}%
\vspace{-4mm}%
\begin{verbatim}
    class POPSynchronizer {
    public:
        POPSyncronizer();
        lock();
        unlock();
        raise();
        wait();
    };
\end{verbatim}
\end{figura}

The \textbf{synchronizer} is an object used for general thread
synchronization inside a parallel object. Every synchronizer has an
associated lock (as in a door lock), and a condition. Locks and
conditions can be used independently of each other or not. The
synchronizer class is presented in the figure \ref{fig:synchro}.

Calls to \texttt{lock()} close the lock and calls to \texttt{unlock()}
open the lock. A call to \texttt{lock()} returns immediately if the lock
is not closed by any other threads. Otherwise, it will pause the execution
of the calling thread until other threads release the lock. Calls to
\texttt{unlock()} will reactivate one (and just one) eventually paused
call to \texttt{lock()}). The reactivated thread will then succeed
closing the lock and the call to \texttt{lock()} will finally return.
Threads that must not run concurrently can exclude each other's
execution using synchronizer locks. When creating a synchronizer, by
default the lock is open. A special constructor is provided to create it
with the lock already closed.

\begin{figura}{fig:ex:lock}{Using the synchronizer lock}%
\vspace{-4mm}%
\begin{verbatim}
    parclass Example {
    private:
         POPSynchronizer syn;
         int counter;
    public:
        int getNext() {
            syn.lock();
            int r = ++ counter;
            syn.unlock;
            return r;
        }
    };
\end{verbatim}
\end{figura}

Conditions can be waited and raised. Calls to \texttt{wait()} cause the
calling thread to pause its execution until another thread triggers the signal
by calling \texttt{raise()}. If the waiting thread possess the lock, it will
automatically release the lock before waiting for the signal. When the signal
occurs, the waiting thread will try to re-acquire the lock that it has previously
released before returning control to the caller. 

Many threads can wait for the same condition. When a thread calls the
method \texttt{raise()}, all waiting-for-signal threads are reactivated at
once. If the lock was closed when the \texttt{wait()} was called, the
reactivated thread will close the lock again before returning from the
\texttt{wait()} call. If other threads calls \texttt{wait()} with the
lock closed, all will wait the lock to be open again before they are
actually reactivated.

The typical use of the synchronizer lock is when many threads can modify
a certain property at the same time. If this modification must be done
atomically, no other thread can interfere before it is finished. The
figure \ref{fig:ex:lock} shows an example of this synchronizer usage.

The typical use of a synchronizer condition is when some thread produces
some information that must be used by another, or in a producer-consumer
situation. Consumer threads must wait until the information is
available. Producer threads must signal that the information is already
available. Figure \ref{fig:ex:cond} is an example that shows the use of
the condition.

\begin{figura}{fig:ex:cond}{Using the synchronizer condition}%
\vspace{-4mm}%
\begin{verbatim}
    parclass ExampleBis {
    private:
        int cakeCount;
        boolean proceed;
        Synchronizer syn;
    public:
        void producer(int count) {
            cakeCount = count;
            syn.lock();
            proceed = true;
            syn.raise();
            syn.unlock();
        }
        void consumer() {
            syn.lock();
            if (!proceed) wait();
            syn.unlock();
            /* can use cakeCount from now on... */
        }
    };
\end{verbatim}
\end{figura}



\subsection{Exceptions} \label{sec:exceptions}


Errors can be efficiently handled using exceptions. Instead of handling
each error separately based on an error code returned by a function
call, exceptions allow the programmer to filter and centrally manage
errors trough several calling stacks. When an error is detected inside a
certain method call, the program can throw an exception that will be
caught somewhere else.

The implementation of exceptions in non-distributed applications, where
all components run within the same memory address space is fairly
simple. The compiler just need to pass a pointer to the exception from
the place where it is thrown to the place where it is caught. However,
in distributed environments where each component is executed in a
separate memory address space (and eventually data are represented
differently due to heterogeneity), the propagation of exception back to
a remote component is complex.

\figureps{fig_exception}{130mm}
{Exception handling example}



POP-C++ supports transparent exception propagation. Exceptions thrown in
a parallel object will be automatically propagated back to the remote
caller (figure \ref{fig_exception}). The current POP-C++ prototype allows
the following types of exceptions:

\begin{petitem}

\item Scalar data (\texttt{int}, \texttt{float}, etc.)

\item Parallel objects

\item Objects of class \texttt{POPException} (system exception)

\end{petitem}

All other C++ types (\texttt{struct}, \texttt{class}, vectors) will be
converted to \texttt{POPException} with the \texttt{UNKNOWN}
exception code.

The invocation semantics of POP-C++ affect the propagation of
exceptions. For the moment, only synchronous methods can propagate the
exception. Asynchronous methods will not propagate any exception to the
caller. POP-C++ current behavior is to abort the application execution
when such exception occurs.

Besides the exceptions created by programmers, POP-C++ uses an
exception of type \texttt{POPException} to notify the user about
the following system failure:

\begin{petitem}

\item Parallel object creation fails. It can happen due to the
      unavailability of suitable resources, an internal error on POP-C++
      services, or the failures on executing the corresponding object
      code.

\item Parallel object method invocation fails. This can be due to the network
      failure, the remote resource down, or other causes.

\end{petitem}

The interface of \texttt{POPException} is described bellow:

\begin{verbatim}
class POPException
{
public:
   const paroc_string Extra()const;
   int Code()const;
   void Print()const;
};
\end{verbatim}

\texttt{Code()} method returns the corresponding error code of the exception.
\texttt{Extra()} method returns the extra information about the place where the
exception occurs. This extra information can contains the parallel object name
and the machine name where the object lives.
\texttt{Print()} method prints atext describing the exception.

All exceptions that are parallel objects are propagated by reference.
Only the interface of the exception is sent back to the caller. Other
exceptions are transmitted to the caller by value.



%%%%%%%%%%%%%%%%%%%%%%%%%%%%%%%%%%%%%%%%%%%%%%%%%%%%%%%%%%%%%%%%%%%%%%%%%%%
\section{Coupling MPI code} \label{sec:mpi}
%%%%%%%%%%%%%%%%%%%%%%%%%%%%%%%%%%%%%%%%%%%%%%%%%%%%%%%%%%%%%%%%%%%%%%%%%%%

POP-C++ can encapsulate MPI processes in parallel objects, allowing POP-C++ applications to use existing HPC MPI libraries. Each MPI process will become a parallel object in POP-C++. The user can control the MPI-based using:

\begin{petitem}

\item Standard POP-C++ remote method invocations. This allows the user to initialize data or computation on some or all MPI processes. 

\item MPI communication primitives such as \texttt{MPI\_Send}, \texttt{MPI\_Recv}, etc. These primitives will use vendor specific communication protocol (e.g. Myrinet/GM).  

\end{petitem}


Each MPI process in POP-C++ will become a parallel object of identical type that can be accessed from outside through remote method invocations.

\begin{figura}{fig:mpi_obj}{MPI parallel objects}%
\vspace{-4mm}%
\begin{verbatim}
parclass TestMPI {
public:
  TestMPI();
  async void ExecuteMPI();
  async void Set(int v);
  sync void Get();
private:
   int val;
};

TestMPI::TestMPI() {
  val=0;
}

void TestMPI::ExecuteMPI() {
  MPI_Bcast(&val,1,MPI_INT, 0, MPI_COMM_WORLD);
}

void TestMPI::Set(int v) {
  val=v;
}
int TestMPI::Get() {
  return val;
}

\end{verbatim}
\end{figura}

Figure \ref{fig:mpi_obj} shows an example of using MPI in POP-C++. \texttt{TestMPI} methods contains some MPI code. Users need to implement a method named \texttt{ExecuteMPI}. This method is invoked on all MPI processes. In this case, the method will broadcast the local value \texttt{val} of process 0 to all other processes.

\begin{figura}{fig:mpi_main}{Creating MPI parallel objects}%
\vspace{-4mm}%
\begin{verbatim} 
#include <popc_mpi.h>
int main(int argc, char **argv) {
  POPMPI<TestMPI> mpi(2);
  mpi[0].Set(100); //Set on MPI process 0
  printf(``Values before: proc0=%d, proc1=%d\n'', 
                         mpi[0].Get(), mpi[1].Get());
  mpi.ExecuteMPI(); //Call ExecuteMPI methods on all MPI processes
  printf(``Values after: proc0=%d, proc1=%d\n'', 
                         mpi[0].Get(), mpi[1].Get());
}

----------
Output of the program:
Values before: proc0=100, proc1=0
Values after: proc0=100, proc1=100

\end{verbatim}
\end{figura}
   

Since an MPI program requires special treatment at startup (mpirun, 
MPI\_Initialize, etc.), users must use a POP-C++ built-in class 
template \texttt{POPMPI} to create parallel object-based MPI processes. 
Figure \ref{fig:mpi_main} illustrates how to start and to call MPI processes. 
We first create 2 MPI processes of type \texttt{TestMPI} using the template 
class \texttt{POPMPI} (variable \texttt{mpi}). Then we can invoke methods 
on a specific MPI process using its rank as the index. \texttt{ExecuteMPI} 
is a pre-defined method of \texttt{POPMPI} which will then invokes all 
corresponding \texttt{ExecuteMPI} methods of the MPI parallel objects 
(\texttt{TestMPI}).


The declaration of \texttt{POPMPI} is described as follows:

\begin{verbatim} 
template<class T> class POPMPI
{
 public:
  POPMPI(); //Do not create MPI process 
  POPMPI(int np); // Create np MPI process of type T
  ~POPMPI(); 

  bool Create(int np); // Create np MPI process of type T
  bool Success(); // Return true if MPI is started. 
                  // Otherwise, return false
  int GetNP(); //Get number of MPI processes

  bool ExecuteMPI(); //Execute method ExecuteMPI on all processes
  inline operator T*(); //type-cast to an array of parclass T
};
\end{verbatim}


%%%%%%%%%%%%%%%%%%%%%%%%%%%%%%%%%%%%%%%%%%%%%%%%%%%%%%%%%%%%%%%%%%%%%%%%%%%
\section{Limitations}
%%%%%%%%%%%%%%%%%%%%%%%%%%%%%%%%%%%%%%%%%%%%%%%%%%%%%%%%%%%%%%%%%%%%%%%%%%%


There are certain limitations to the current implementation of POP-C++. 
Some of these restrictions are expected to disappear in the future while others are simply due to the nature of parallel programming and the impossibility for parallel objects to share a common memory. For the current version (1.3), the limitations are:

\begin{petitem}

\item A parallel class cannot contain public attributes.

\item A parallel class cannot contain a class attribute (\texttt{static}).

\item A parallel class cannot be template.

\item An asynchronous method cannot return a value and cannot have
	output parameters.

\item Global variables exist only in the scope of parallel objects
	(\texttt{@pack()} scope).

\item The programmer must specify the size of pointer parameters in remote method invocation as they are considered as arrays.

\item A parallel object method cannot return a memory address.

\item Sequential classes used as parameter must be derived from
	\texttt{POPBase} and the programmer must implement the
	\texttt{Serialize} method.

\item Parameters must have exactly the same dynamic type as in method
	declaration, an object of a derived class cannot be used (polymorphism).

\item Exceptions. Only scalar, parallel object and POPException type are handled. All other exceptions are converted to POPException with the {\it unknown} code.

\item Exceptions raised in an asynchronous method are not propagated. They abort
(cleanly) the application.

\end{petitem}
