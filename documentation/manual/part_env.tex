\chapter{Runtime environment variables}

The following environment variables affect or change the default behaviors of the POP-C++ runtime. To ensure that the environment of all running objects these variables should all be set during the installation \texttt{make install} or in the environment setup script \texttt{popc-runtime-env}.

\begin{verbatim} 
POPC_LOCATION:         Location of installed POP-C++ directory.

POPC_PLUGIN_LOCATION:  Location where additional communication
                        and data encoding plugins can be found. 

POPC_JOBSERVICE:       The access point of the POP-C++ job manager 
                        service. If the POP-C++ job manager does not
                        run on the local machine where the user start 
                        the application, the user must explicitly 
                        specify this information. 
                        Default value: socket://localhost:2711. 

POPC_HOST:             Full qualified host name of local node. 
                       This host name will be interpreted

POPC_IP:               IP of local node. Only used if POPC_HOST is
                        not defined

POPC_IFACE:            If POPC_HOST and POPC_IP are not set, use
                        this interface to determine node IP. If not
                        set, the default gateway interface is used.

POPC_PLATFORM:         The platform name of the local host. 
                        By default, the following format is used:
                        <cpu id>-<os vendor>-<os name>.

POPC_MPIRUN:           The mpirun command to start POP-C++ MPI 
                        objects.

POPC_JOB_EXEC:         Script used by the job manager to submit 
                        a job to local system. 

POPC_DEBUG:            Print all debug information. 

\end{verbatim} 
