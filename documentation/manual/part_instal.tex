\chapter{Installation Instructions}

%%%%%%%%%%%%%%%%%%%%%%%%%%%%%%%%%%%%%%%%%%%%%%%%%%%%%%%%%%%%%%%%%%%%%%%%%%%
\section{Before installing}
%%%%%%%%%%%%%%%%%%%%%%%%%%%%%%%%%%%%%%%%%%%%%%%%%%%%%%%%%%%%%%%%%%%%%%%%%%%

To find out about the latest sources releases or installation instructions please visite our wiki :

\texttt{http://gridgroup.hefr.ch/popc}

POP-C++ is built on top of several widely known software packages. The following packages 
of are required before compiling.

\begin{petitem}

\item a C++ compiler (g++)

\item zlib-devel

\item the Gnu Bison (optional)

\item the Globus Toolkit (optional)
\end{petitem}

Before installation we should make the following configuration choices. 
In case of doubt the default values can be used.

\begin{petitem}

\item The compilation directory that should hold roughly 50MB. This
directory will contain the distribution tree and the source files of
POP-C++. It may be erased after installation.

\item The installation directory that will hold less than 40MB. It will
contain the compiled files for POP-C++, include and configuration files.
This directory is necessary in every computer executing POP-C++
programs. (by default \texttt{/usr/local/popc})

\item A temporary directory will be asked in the installation process.
This directory will be used by POP-C++ to hold files during the
applications execution. (by default \texttt{/tmp})

\item Resource topology. The administrator must choose what computers form our grid.
\end{petitem}

%%%%%%%%%%%%%%%%%%%%%%%%%%%%%%%%%%%%%%%%%%%%%%%%%%%%%%%%%%%%%%%%%%%%%%%%%%%
\section{Standard Installation}
%%%%%%%%%%%%%%%%%%%%%%%%%%%%%%%%%%%%%%%%%%%%%%%%%%%%%%%%%%%%%%%%%%%%%%%%%%%

This section explains how to install POP-C++ using all default options. This is usually sufficient 
if you want to test POP-C++ on your desktop computer. An advanced installation is explained in the two sections below.

POP-C++ distribution uses standard GNU tools to compile. The following
commands will generate all necessary files:

\texttt{~\\
~~~~cd }\textit{compilation-directory}\texttt{\\
~~~~tar xzf popc-}\textit{version}\texttt{.tar.gz\\
~~~~cd popc-}\textit{version}\texttt{\\
~~~~./configure\\
~~~~make} \\
~

The \texttt{make} command takes a while to finish. After it, all files
will be compiled and POP-C++ is ready for installation. To install
POP-C++ type:

\texttt{~\\
~~~~make install} \\
~

After copying the necessary files to the chosen installation directory,
a setup script is run. It asks different questions, and the information
gathered before the installation should be sufficient to answer it.

For a standard installation of POP-C++ it is sufficient to pick the simple installation when asked.

In case it is necessary to restart the setup script, it can be done
with the following command:

\textit{~\\
~~~~~installation-directory}\texttt{/sbin/popc\_setup}\\
~


%%%%%%%%%%%%%%%%%%%%%%%%%%%%%%%%%%%%%%%%%%%%%%%%%%%%%%%%%%%%%%%%%%%%%%%%%%%
\section{Custom Installation}
%%%%%%%%%%%%%%%%%%%%%%%%%%%%%%%%%%%%%%%%%%%%%%%%%%%%%%%%%%%%%%%%%%%%%%%%%%%

The configuration utility can be started with command line arguments
for a custom installation. These arguments control whether some extra or
different features should be enabled. A list of optional features can be
found in the figure \ref{fig:features}. The full list of options
accepted by the configuration utility can be obtained with the
\texttt{---help} argument.

\begin{figura}{fig:features}{Optional configuration features}
\begin{tabular}{| l | l |}
\hline
%\tt --disable-FEATURE    		& do not include FEATURE \\
%\tt 					& (same as --enable-FEATURE=no)\\ \hline
%{\tt --enable-FEATURE}$[${\tt =}$ARG]$  & include FEATURE ($ARG$=yes)\\ \hline
\tt -\--enable-mpi  	& Enable MPI-support in POP-C++\\ \hline
\tt -\--enable-globus=flavor  	& Enable Globus-support in POP-C++\\ \hline
\tt -\--enable-xml  			& Enable XML encoding in POP-C++\\ \hline
%\tt --enable-myrinet  		& Enable GM/Myrinet communication\\ \hline
%\tt 					& protocol in POP-C++\\ \hline
\tt -\--enable-http  			& Enable HTTP communication protocol in POP-C++\\ \hline
%{\tt --enable-shared}$[${\tt =}$PKGS]$ 	& build shared libraries (default=yes)\\ \hline
%{\tt --enable-static}$[${\tt =}$PKGS]$	& build static libraries (default=yes)\\ \hline
%{\tt --enable-fast-install}$[${\tt =}$PKGS]$ & optimize for fast installation (default=yes)\\ \hline
%\tt --disable-libtool-lock  		& avoid locking (might break parallel builds)\\ \hline
\end{tabular}
\end{figura}


The current distribution of POP-C++ 1.3 supports the following features:
\begin{petitem}

\item Globus-enabled services. POP-C++ allows to build the runtime
services for Globus. We only use the Pre WS GRAM of Globus Toolkit
(GT3.2 or GT4).  To enable this feature, you will need to provide the
Globus's built flavor (refer Globus documentation for more information).
Before configuring POP-C++  with Globus, you need to set the environment
variable GLOBUS\_LOCATION to the Globus installed directory. Bellow is
an example of configuring POP-C++ for Globus with the flavor of
\texttt{gcc32dbgpthr}:

./configure -\--enable-globus=gcc32dbgpthr

\item Enable SOAP/XML encoding. POP-C++ supports multiple data encoding
methods as the local plugins to the applications. POP-C++ requires the
Xerces-C library to enable SOAP/XML encoding (configure --enable-xml).

\item Enable HTTP protocol in parallel object method invocations. This
protocol allows objects to communicate cross sites over the firewall
(experimental feature).    

\item Enable MPI support. This feature allows POP-C++ applications to implement
parallel objects as MPI processes (refer section \ref{sec:mpi}). 

\end{petitem}

%%%%%%%%%%%%%%%%%%%%%%%%%%%%%%%%%%%%%%%%%%%%%%%%%%%%%%%%%%%%%%%%%%%%%%%%%%%
\section{Configuring POP-C++ services}
%%%%%%%%%%%%%%%%%%%%%%%%%%%%%%%%%%%%%%%%%%%%%%%%%%%%%%%%%%%%%%%%%%%%%%%%%%%

The POP-C++ runtime service is a fully distributed model where resource connectivity is represented as a
dynamic graph. A resource (POP-C++ service node) can join the
environment  by registering itself to a node (or a master) inside this
environment (dynamic) or by being listed statically in the ``known
nodes'' of other resources inside the environment. 

When configuring POP-C++ services on each node, the user will be prompted to give information
about the master nodes (to which the configuring POP-C++ service will
register itself to) and about the child nodes that the configuring
POP-C++ service will manage. 

\begin{petitem}
\item The number of processors available on the resource (node).
If the POP-C++ service represents a
front end of a cluster, the number of processors is the number of nodes
of that cluster. In this case, you will need to specify the script to
submit a job to the cluster.  

\item The local username to be used in the child nodes, in the case
POP-C++ is started by the \texttt{root} user (this item is optional).

\item The TCP/IP port to be used by POP-C++ (this item is optional, by
default the port 2711 is used).

\item The domain name of the local resource (optional). If no domain is
provided, the IP address will be used. 


Please note that more option can be set in appendix B : Runtime environment
variables.


\end{petitem}


When you run ``make install'' for the first time, it will automatically execute the following script:

 \textit{~\\
installation-directory}\texttt{/sbin/popc\_setup}\\
~

This script will ask you several question about the local resource and the execution environment. 


We assume to configure POP-C++ on 25 workstations sb01.eif.ch-sb025.eif.ch. We choose the machine sb02.eif.ch as the master node and the rest will register to this machine upon starting the POP-C++ services. We configured POP-C++ with Globus Toolkit 4.0. The POP-C++installation is shared on NFS among all machines.  Following is a transcript:



\begin{enumerate}

    \item Configure POP-C++ service on your local machine:

POP-C++ runtime environment assumes the resource topology is a graph.
Each node can join the environment by register itself to other nodes
(master hosts). If you want to deploy POP-C++ services at your site, you
can select one or several machines to be master nodes and when configure
other nodes, you will need to enter these master nodes as requested.
Another possibility to create your resource graph is to explicitly
specify list of child nodes to which job requests can be forwarded to.
Here is an example:

\begin{verbatim}
------------------
Enter the full qualified master host name (POPC gateway):
sb02.eif.ch

Enter the full qualified master host name (POPC gateway):
[Enter]

Enter the child node:
[Enter]
------------------
\end{verbatim}

   \item Information of the local execution environment:

   \begin{petitem}

	\item Number of processors of the local machine. If you intend to
	run POP-C++ service on a front end of a cluster, this can be the number
	of nodes inside that cluster.

	\item Maximum number of jobs that can be submitted to your local
	machine.

	\item The local user account you would like to run jobs. This is
	only applied to the standalone POP-C++ services. In the case you use
	Globus to submit jobs, authentication and authorization are provided
	by Globus, hence, this information will be ignored.

	\item Environment variables: you can set up your environment
	variables for your jobs. Normally, you need to set the
	\textbf{LD\_LIBRARY\_PATH} to all locations where dynamic libraries are
	found. 

   \end{petitem}

	\item If you enable Globus while configuring POP-C++, information
	about Globus environment will be prompted:

    \begin{petitem}

	\item The Globus gatekeeper contact: this is the host
	certificate of the local machine. If you intend to share the
	same Globus host certificate among all machines of your site,
	you should  provide this certificate here instead of the
	Globus's gatekeeper contact.

	\item Globus grid-mapfile: POP-C++ will need information from
	the Globus's grid-mapfile to verify if the user is eligible for running
	jobs during resource discovery.

    \end{petitem}

Here is an example of what you will be asked:
\begin{small}
\begin{verbatim}
----------------
Enter number of processors available (default:1):
[Enter]

Enter the maximum number of POP-C++ jobs that can run
concurrently(default: 1):
[Enter]

Which local user you want to use for running POPC jobs?
[Enter]

CONFIGURING THE RUNTIME ENVIRONMENT

Enter the script to submit jobs to the local system:
[Enter]

\end{verbatim}

%NOTE: this information is required if you use a batch job management system. 
%                An example of PBS script is provided  in services/popcjob.pbs of the installation tree.


\begin{verbatim}
Communication pattern:
\end{verbatim}

NOTE: Communication pattern is a text string defining the protocol priority on binding the interface to the object server. It can contain ``*'' (matching non or all) and ``?'' (matching any) wildcards. 

For example: given communication pattern ``\texttt{socket://160.98.* http://*}'':
\begin{petitem}
\item If the remote object access point is

 ``\texttt{socket://128.178.87.180:32427 http://128.178.87.180:8080/MyObj}'', 

the protocol to be used will be ``http''.

\item If the remote object access point is 

``\texttt{socket://160.98.20.54:33478 http://160.98.20.54:8080/MyObj}'', 

the protocol to be used will be ``socket''.

\end{petitem}
 
\begin{verbatim}
SETTING UP RUNTIME ENVIRONMENT VARIABLES

Enter variable name:
LD_LIBRARY_PATH

Enter variable value:
/usr/openwin/lib:/usr/lib:/opt/SUNWspro/lib

Enter variable name:
[Enter]

DO YOU WANT TO CONFIGURE POPC SERVICES FOR GLOBUS? (y/n)
y

Enter the local globus gatekeeper contact:
/O=EIF/OU=GridGroup/CN=host/eif.ch

Enter the GLOBUS grid-mapfile([/etc/grid-security/grid-mapfile]):
[Enter]

=====================================================
CONFIGURATION POP-C++ SERVICES COMPLETED!
=====================================================
---------------
\end{verbatim}
\end{small}

    \item Generate startup script: you will be asked to generate startup scripts for POP-C++ services. These scripts (SXXpopc*) will be stored in the sbin subdirectory of the POP-C++ installed directory.

\begin{petitem}

	\item The local port where POP-C++ service is running. It is
	recommended to keep the default port (2711).

	\item The domain name of the local host. If your machine is not
	listed in the DNS, just leave this field empty.

	\item Temporary directory to store log information. If you leave
	this field empty, /tmp will be used.

	\item If you configure POP-C++ with Globus, the Globus installed
	directory will also been prompted.

\end{petitem}

Bellow is the example:

\begin{verbatim}
---------------
Do you want to generate the POPC++ startup scripts? (y/n)
y
=====================================================
CONFIGURING STARTUP SCRIPT FOR YOUR LOCAL MACHINE...
Enter the service port[2711]:
[Enter]

Enter the domain name:
eif.ch

Enter the temporary directory for intermediate results:
/tmp/popc

DO YOU WANT TO GENERATE THE GLOBUS-BASED POPC SCRIPT? (y/n)
y

Enter the globus installed directory (/usr/local/globus-4.0.0):
[Enter]

CONFIGURATION DONE!
---------------
\end{verbatim}

\end{enumerate}

If you want to change the POP-C++ configuration, you can manually run
the configure script \textbf{popc\_setup} located in the
\emph{$<$installed directory$>$/sbin}




%%%%%%%%%%%%%%%%%%%%%%%%%%%%%%%%%%%%%%%%%%%%%%%%%%%%%%%%%%%%%%%%%%%%%%%%%%%
\section{System Setup and Startup}
%%%%%%%%%%%%%%%%%%%%%%%%%%%%%%%%%%%%%%%%%%%%%%%%%%%%%%%%%%%%%%%%%%%%%%%%%%%

The installation tree provides a shell setup script. It sets
paths to the POP-C++ binaries and library directories. The most
straightforward solution is to include a reference to setup script in
the users login shell setup file (like \texttt{.profile}, \texttt{.bashrc} or
\texttt{.cshrc}). The setup scripts (respectively for C-shells and
Bourne shells) are:

\textit{~\\
installation-directory}\texttt{/etc/popc-user-env.csh } and\\
\textit{installation-directory}\texttt{/etc/popc-user-env.sh}\\
~

Before executing any POP-C++ application, the runtime system (job manager) must be
started. There is a script provided for that purpose, so every node
must run the following command:

\textit{~\\
installation-directory}\texttt{/sbin/SXXpopc start}\\
~

\texttt{SXXpopc} is a standard Unix daemon control script, with the
traditional \texttt{start}, \texttt{stop} and \texttt{restart} options.
There is a different version to be used with Globus, called
\texttt{SXXpopc.globus}.
